\textbf{5)} 

A ideia de usar a câmera como ponto de partida 
para a renderização de raycast é simples. Para
otimizar o custo de processamento do computador 
na construção da cena, a partir de uma referência 
do que será mostrado na tela final, o feixe da 
câmera é emitido para a frente até que o primeiro 
objeto seja encontrado e construído na cena. Desta
forma, objetos atrás do objeto em colisão ou 
objetos que não aparecerão na imagem final não 
serão construídos.
