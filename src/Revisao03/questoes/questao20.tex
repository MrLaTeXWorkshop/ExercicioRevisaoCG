\textbf{20)} 

Na Cinemática Direta, entramos com a posição inicial(em um tempo qualquer), a 
velocidade e aceleração de um objeto, e queremos como saída 
a posição nos próximos tempos.

Já na Cinemática Inversa, incluimos a posição inicial(e seu tempo),
e então colocamos a segunda posição(e seu tempo), com isso, a saída 
será a velocidade e aceleração necessária para completar essa atravessia.


