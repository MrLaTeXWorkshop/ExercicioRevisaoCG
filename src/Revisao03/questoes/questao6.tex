\textbf{6)} 

\begin{itemize}
	\item \textbf{Volumes limítrofes} - Encapsular objetos por regiões com 
	seções planas como retângulos(paralelepípedos), circunferências(esferas),
	elipses(elipsóides). Isto simplifica os cálculos das retas, com as seções do objeto.
	
	\item \textbf{Hierarquia de volumes limítrofes} - Agrupam-se outros volumes dentro de um volume maior.
	O método de agrupamento é arbitrário, pode utilizar a distribuição dos objetos na cena.
	
	\item \textbf{Grid uniforme} - Seção plana da cena divida em uma matriz,
	onde cada célula possuem o mesmo tamanho .

	\item \textbf{Octree} - Aspecto da cena como critério de homogeneidade,
	flexibiliza células(regiões das seções planas) de tamanhos variados
	
	\item \textbf{BSP} - Possibilidade de divisões em regiões mais otimizadas 
	para reduzir volume de cálculos das interseções.
\end{itemize}