\section*{Rasterização de Retas}

	\begin{enumerate}\addtocounter{enumi}{5}
		\item 
		
		O valor de delta refere-se ao dx e dy, ou seja, a diferença entre ao par (x,y) do
		ponto inicial e final da reta. Logo, para atravessar o canvas e preencher os pontos
		da reta, o número de iterações será essa diferença.
		
		\item 
		
       Para o algoritmo do DDA, o motivo de usar apenas valores positivos se dá no
      momento em que o passo de \textbf{x} precisa ser incrementado. Se não usar o valor absoluto nessa equação não será possível desenhar o deslocamento da linha AB, contudo se todo o algoritmo for modificado para usar um passo de \textbf{x} negativo, será possível sim usar valores de delta negativos. Já no caso de Bresenham, é impossível, pois seu algoritmo é bem mais complexo e divide o ângulo da reta em 8 partes, valores negativos implicariam em mudar consideravelmente sua fórmula, uma vez que para evitar deltas negativos no algoritmo de Bresenham, define que deve-se inverter a direção do desenho para que volte a ser usado apenas valores positivos.
	\end{enumerate}