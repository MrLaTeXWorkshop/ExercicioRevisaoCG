\section*{Rasterização de Circunferências}

	\begin{enumerate} \addtocounter{enumi}{15}
		\item 
		O algoritmo projeta um circulo perfeito, logo, ele só precisa calcular apenas um
		octante, e o resto, o algoritmo projeta o mesmo resultado para os outros octantes.
		
		\item 
		O que restringe o desenho da circunferência no segundo quadrante é as linhas:
		\textbf{desenharPixel(xc -x, yc - y)} e \textbf{desenharPixel(xc - y, yc - x)}. 
		Todas as operações de	escrita do pixel quando o x e o y são negativos.		
		
		\item 
		A atualização do 'x' precisa ser atualizado antes da variável 'p' e 'y', pois o cálculo
		da variável de decisão 'p' requer o uso do valor em 'x', que precisa ser atualizado. 
		Enquanto que o 'y', apenas atualiza, se 'p' for maior que zero, justificando a atualização
		de 'x' anterior as outras duas variáveis.
		 
		\item 
       Durante chamada da função de desenhar os 8 pontos do círculos as posições do X e do Y 
       que não estão na origem são utilizados. Dessa forma, durante o algoritmo em si não é 
       necessário saber se o circulo não está na origem, só quando os pontos serão desenhados.
       
		\item
		\begin{enumerate}[label=\alph*. ]
			\item 
		
            x = 2, y = 4, p = 5

                Desenhar círculo:
                
                   xc + x = 1 , yc + y = 6
                   
                   xc - x = -3 , yc + y = 6
                   
                   xc + x = 1 , yc - y = -2
                   
                   xc - x = -3 , yc - y = -2
                   
                   xc + x = 1 , yc + y = 6
                   
                   xc + x = 3 , yc + y = 4
                   
                   xc - x = -5 , yc + y = 4
                   
                  xc + x = 3 , yc - y = 0
                  
                  xc - x = -5 , yc - y = 0
            
            \item 
			 xc = -1, yc = 2, r = 5            
            
            x = 3, y = 3, p = 15

               Desenhando círculo:
               
                  xc + x = 3 , yc + y = 3
                  
                  xc - x = -3 , yc + y = 3
                  
                  xc + x = 3 , yc - y = -3
                  
                  xc - x = -3 , yc - y = -3
                  
                  xc + x = 3 , yc + y = 3
                  
                  xc + x = 3 , yc + y = 3
                  
                  xc - x = -3 , yc + y = 3
                  
                  xc + x = 3 , yc - y = -3
                  
                  xc - x = -3 , yc - y = -3

            \item 
             xc = 3, yc = 4, r = 6
             
             x = 2, y = 5, p = -1
             
             Desenhar círculo:
             
					xc + x = 5 , yc + y = 9
					
					xc - x = 1 , yc + y = 9
					
					xc + x = 5 , yc - y = -1
					
					xc - x = 1 , yc - y = -1
					
					xc + x = 5 , yc + y = 9
					
					xc + x = 8 , yc + y = 6
					
					xc - x = -2 , yc + y = 6
					
					xc + x = 8 , yc - y = 2
					
					xc - x = -2 , yc - y = 2
                
		\end{enumerate}
		
	\end{enumerate}
