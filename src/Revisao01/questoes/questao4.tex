\section*{Bresenham}

	\begin{enumerate}\addtocounter{enumi}{10}
	
		\item O algoritmo de bresenham trabalha com inteiros em vez de pontos flutuantes(floats),
		isso permite com que as linhas sejam mais precisas, quando comparadas ao algoritmo DDA.
		Sem contar o gasto em memória é menor, já que precisamos de menor bytes para armazenar 	       valores inteiros do que pontos flutuantes.
		
		\item 
O algoritmo de Bresenham se torna extremamente eficiente ao dividir os ângulos de
desenho de reta em 8 partes. O uso apenas de valores positivos se dá por conta do passo do
algoritmo. Para se desenhar uma linha deve-se ir desenhando da esquerda para a
direita. O inverso implicaria em modificar drasticamente a implementação do algoritmo para
permitir que linhas possam ser desenhadas da direita para a esquerda, o que só aumentaria a
complexidade desse, uma vez que apenas invertendo os pontos resultaria na mesma linha. Por
exemplo, se um ponto A está a direita de B, não é necessário implementar a escrita da linha da
direita para a esquerda e sim trocar a ordem dos pontos, desenhando de B para A.

		\item 
A ideia do algoritmo em sempre atualizar \textbf{x} antes da variável de controle \textbf{p} é em
sempre começar desenhando naquele pixel para depois ver se será necessário desenhar em um pixel acima ou abaixo desse, essa foi uma escolha de implementação feita por Bresenham. Se a linha que estiver sendo desenhada estiver o \textbf{m} maior que 1, então teremos que atualizar \textbf{y} ao invés de \textbf{x}. E seguir a mesma ideia de posteriormente atualizar a variável  \textbf{p}, para ver se precisamos desenhar em outro pixel em \textbf{x}.

		\item 
		Quando a variável \textbf{p} for positiva, incrementamos o valor do \textbf{y},
		isso acontece pois \textbf{p} é a variável de decisão que irá conduzir ao algoritmo
		qual pixel ele deve tomar na hora de formar a linha. Como \textbf{p} é positivo, 
		significa que a diferença entre d1 e d2(o pixel anterior e posterior) é maior que zero,
		por isso, precisamos que o valor de y incremente.

        \item \begin{enumerate}[label=\alph*.]
				   \setlength\itemsep{1em}
					\item	 AB – A(-1,4) e B(5, 7)
					   				
                       -1, 4
                       
                        0, 5
                        
                        1, 5
                        
                       2, 6
                       
                       3, 6
                       
                       4, 7
                       
                        5, 7
					
					\item    BA – B(5, 7) e A(-1, 4)
					
                           5, 7
                           
                          4, 6
                          
                          3, 6
                          
                          2, 5
                          
                          1, 5
                          
                         0, 4
                         
                         -1, 4
									
					\item  CD – C(-1, 4) e D(3, 8)
					
						 -1, 4
						 
						 0, 5
						  
						 1, 6
						  
						 2, 7
						  
						 3, 8
									
					\item EF – E(2, 0) e F(6, 0)
					
						2, 0
						
						3, 0
						
						4, 0
						
						5, 0
						
						6, 0
									
					\item   GH – G(1, 3) e H(1, 6)
									
                      1, 3
                      
                      1, 4
                      
                      1, 5
                      
                      1, 6
									
				\end{enumerate}
	\end{enumerate}