\section*{Bresenham}

\begin{question}

	\begin{enumerate}\addtocounter{enumi}{10}
	
		\item O algoritmo de bresenham trabalha com inteiros em vez de pontos flutuantes(floats),
		isso permite com que as linhas sejam mais precisas, quando comparadas ao algoritmo DDA
		
		\item 

		\item 

		\item 

        \item \begin{enumerate}[label=\alph*.]
				   \setlength\itemsep{1em}
					\item	 AB – A(-1,4) e B(5, 7)
					   				
                       -1, 4
                       
                        0, 5
                        
                        1, 5
                        
                       2, 6
                       
                       3, 6
                       
                       4, 7
                       
                        5, 7
					
					\item    BA – B(5, 7) e A(-1, 4)
					
                           5, 7
                           
                          4, 6
                          
                          3, 6
                          
                          2, 5
                          
                          1, 5
                          
                         0, 4
                         
                         -1, 4
									
					\item  CD – C(-1, 4) e D(3, 8)
					
						 -1, 4
						 
						 0, 5
						  
						 1, 6
						  
						 2, 7
						  
						 3, 8
									
					\item EF – E(2, 0) e F(6, 0)
					
						2, 0
						
						3, 0
						
						4, 0
						
						5, 0
						
						6, 0
									
					\item   GH – G(1, 3) e (1, 6)
									
                      1, 3
                      
                      1, 4
                      
                      1, 5
                      
                      1, 6
									
				\end{enumerate}
	\end{enumerate}
	
\end{question}
