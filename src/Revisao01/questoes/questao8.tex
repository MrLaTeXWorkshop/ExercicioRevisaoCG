\section*{Liang-Barsky}

	\begin{enumerate}\addtocounter{enumi}{27}
	
		\item 
		Apenas no final é feito a atualização, pois primeiro é preciso verificar se a reta está dentro
		da área de clipping, mas também é preciso verificar qual é o ajuste é necessário para 
		que a reta fique dentro da área desejada
		
		\item 
		As estruturas condicionais são aninhadas, pois o programa verifica pelos cantos da
		área de clipping, ou seja, os cantos direito e esquerdo, tanto inferior e superior.
		\item 
		u1 e u2 são parámetros de interseção da linha, quando u1 é maior que u2, rejeitamos
		essa linha, pois significa que ela não está mais dentro da área de clipping, logo, iniciamos
		u1 com 0 e u2 com 1, pois 0 < 1.
		\item
		
		Primeira Linha(AB) $\rightarrow$ (-1,-3) $\rightarrow$ (-2,-8)
		
		dy = 11.0

		Linha rejeitada

		Segunda linha (BC) $\rightarrow$ (-2,-8) $\rightarrow$ (9,2)

		dy = -6.0
       t2 = 0.72
       t1 = 0.33

		Linha aceita: 
		
 		x1: 1 ,x2: 6 ,y1: 6 ,y2: 3
 		
 		Segunda linha (AC) $\rightarrow$ (-1,-3) $\rightarrow$ (9,2)

		dy =5.0
       t2 = 0.7
       t1 = 0.6

		Linha aceita: 
		
		x1: 5 ,x2: 6 ,y1: 0 ,y2: 0 
		
	\end{enumerate}
