\section*{DDA}

	\begin{enumerate}\addtocounter{enumi}{7}
		\item 
		Os valores são arredondados apenas na visualização, pois o posicionamento
		dos pixeis são representados por número inteiros. Como o algoritmo DDA trabalha
		com pontos flutuantes, precisamos arrendondar esses valores na hora de exibí-los no
		canvas
		\item 
       Durante todo o algoritmo do DDA, é importante manter os valores em ponto flutuante
       para manter-se a maior precisão possível do formato da linha que será desenhada na tela. A
       transformação para número inteiro se dá por conta dos  displays  que usamos serem  construídos apenas com números inteiros de pixels. Não sendo possível desenhar em uma tela em uma posição fracionada.
		\item 
		
		\begin{enumerate}[label=\alph*.]
				   \setlength\itemsep{1em}
					\item	 AB – A(-1,4) e B(5, 7)
					   				
                         -1, 4
                         
                          0, 5
                          
                           1, 5
                           
                           2, 6
                           
                           3, 6
                           
                           4, 7
                           
                            5, 7
					
					\item    BA – B(5, 7) e A(-1, 4)
					
                           5, 7
                           
                          4, 7
                          
                          3, 6
                          
                          2, 6
                          
                          1, 5
                          
                         0, 5
                         
                         -1, 4
									
					\item  CD – C(-1, 4) e D(3, 8)
					
						 -1, 4
						 
						 0, 5
						  
						 1, 6
						  
						 2, 7
						  
						 3, 8
									
					\item EF – E(2, 0) e F(6, 0)
					
						2, 0
						
						3, 0
						
						4, 0
						
						5, 0
						
						6, 0
									
					\item   GH – G(1, 3) e H(1, 6)
									
                      1, 3
                      
                      1, 4
                      
                      1, 5
                      
                      1, 6
									
				\end{enumerate}
	
	\end{enumerate}