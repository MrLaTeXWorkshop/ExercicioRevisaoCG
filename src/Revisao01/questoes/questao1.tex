\section*{Transformações Geométricas}

	\begin{enumerate}[label=\arabic*)]
	    \setlength\itemsep{1em}
	    
		\item Coordenadas homogêneas permite o tratamento algébrico de pontos
	   no infinito.
		
		\item 

		\item 
		Essa movimentação só pode ser evitada, garantindo que todo objeto tenha um 
		ponto central, dessa forma, para fazer as operações, o vetor para escalonar é a diferença do
		vértice e esse ponto central.

		\item 

		\item 
				
				\begin{enumerate}[label=\alph*.]
				   \setlength\itemsep{1em}
					\item					
					   				
					   				A(-1,-3) $\rightarrow$ A'(-2,2);
				
									B(-2,8)  $\rightarrow$ B'(-3,13);
				
									C(9,2)  $\rightarrow$ C' (8,7);
					\item 
					
								    A(-1,-3) $\rightarrow$ A'(-2.36,-2.09);
				
									B(-2,8)  $\rightarrow$ B'(2.26,7.92);
				
									C(9,2)  $\rightarrow$ C' (8.79,-2.76);
					\item 
					
									A(-1,-3) $\rightarrow$ A'(-2.09,-2.36);
				
									B(-2,8)  $\rightarrow$ B'(-2,8);
				
									C(9,2)  $\rightarrow$ C' (2.76,8.79);
					\item
					
									A(-1,-3) $\rightarrow$ A'(-0.5,-6);
				
									B(-2,8)  $\rightarrow$ B'(-1,24);
				
									C(9,2)  $\rightarrow$ C' (4.5,21);
					\item
									
									A(-1,-3) $\rightarrow$ A'(1,-3);
				
									B(-2,8)  $\rightarrow$ B'(2,8);
				
									C(9,2)  $\rightarrow$ C' (-9,2);
								
				\end{enumerate}

	\end{enumerate}
	
