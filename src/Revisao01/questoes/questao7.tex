\section*{Cohen-Sutherland}

	\begin{enumerate}\addtocounter{enumi}{21}
		\item 
		
		\item 
		Há duas condições de paradas para o algoritmo, no primeiro caso, ambos os pontos
		estão dentro da área de clipping, no outro caso, ambos os pontos estão fora da área;
		\item
		
		\item 
		c1 e c2 são as váriaveis que vão computar a posição dos pontos da reta
		inserida no algoritmo. Ambas são números onde cada bit representa a direção
		 que o ponto está, em correspondência a área de clipping. O resultado dessas váriaveis
		 serão armazenados e então é feito a operação descrita "c1 \& c2 != 0", que irá fazer um 
		 bitwise and. Qualquer número diferente de 0 indicará que a computação de c1 e c2 resultou
		 em um número que está fora da área de clipping.
		
		\item 
		
		\item 

		Primeira Linha(AB) $\rightarrow$ (-1,-3) $\rightarrow$ (-2,-8)
		
		Linha calculada: 
		
		 x1: -1 ,x2: -2 ,y1: -3 ,y2: 8

		Linha rejeitada

		Segunda linha (BC) $\rightarrow$ (-2,-8) $\rightarrow$ (9,2)

		Linha calculada: 
		
 		x1: -2 ,x2: 9 ,y1: 8 ,y2: 2

		Linha calculada: 
		
		 x1: 1 ,x2: 9 ,y1: 6 ,y2: 2

		Linha aceita: 
		
 		x1: 1 ,x2: 6 ,y1: 6 ,y2: 3
 		
 		Segunda linha (AC) $\rightarrow$ (-1,-3) $\rightarrow$ (9,2)

		Linha calculada: 
		
 		x1: -1 ,x2: 9 ,y1: -3 ,y2: 2

		Linha calculada: 
		
		x1: 5 ,x2: 9 ,y1: 0 ,y2: 2

		Linha aceita: 
		
		x1: 5 ,x2: 6 ,y1: 0 ,y2: 0
		
	\end{enumerate}
