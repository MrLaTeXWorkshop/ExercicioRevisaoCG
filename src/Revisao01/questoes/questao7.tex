\section*{Cohen-Sutherland}

	\begin{enumerate}\addtocounter{enumi}{21}
		\item 
Os códigos 3 e 7, não são considerados, pois em representação binária seriam 11 e
111 e o primeiro bit representa um lado do recorte o segundo bit o outro lado – o terceiro bit
representa que está em baixo -. Dessa forma, é impossível que eles estejam juntos, não é possível que um ponto de uma linha esteja ao mesmo tempo a direita e a esquerda da janela de clipping.		
		\item 
		Há duas condições de paradas para o algoritmo, no primeiro caso, ambos os pontos
		estão dentro da área de clipping, no outro caso, ambos os pontos estão fora da área;
		\item
Uma das condições de parada é quando a linha não está desenhada dentro da área de
recorte. Outra razão de parada é se a linha estiver inteiramente dentro da área de clipping. Por fim, as outras razões de parada estão relacionadas à encontrar o ponto de intersecção entre a linha e a janela de recorte, depois de descobrir esse ponto e executar novamente para descobrir o outro ponto (se já estiver dentro já podemos parar), então o algoritmo.
		\item 
		c1 e c2 são as váriaveis que vão computar a posição dos pontos da reta
		inserida no algoritmo. Ambas são números onde cada bit representa a direção
		 que o ponto está, em correspondência a área de clipping. O resultado dessas váriaveis
		 serão armazenados e então é feito a operação descrita "c1 \& c2 != 0", que irá fazer um 
		 bitwise and. Qualquer número diferente de 0 indicará que a computação de c1 e c2 resultou
		 em um número que está fora da área de clipping.
		
		\item 
		Para os valores originais da cena antes da execução do código, se não forem salvos
pelo programador antes de executar o algoritmo, serão todos perdidos.
		\item 

		Primeira Linha(AB) $\rightarrow$ (-1,-3) $\rightarrow$ (-2,-8)
		
		Linha calculada: 
		
		 x1: -1 ,x2: -2 ,y1: -3 ,y2: 8

		Linha rejeitada

		Segunda linha (BC) $\rightarrow$ (-2,-8) $\rightarrow$ (9,2)

		Linha calculada: 
		
 		x1: -2 ,x2: 9 ,y1: 8 ,y2: 2

		Linha calculada: 
		
		 x1: 1 ,x2: 9 ,y1: 6 ,y2: 2

		Linha aceita: 
		
 		x1: 1 ,x2: 6 ,y1: 6 ,y2: 3
 		
 		Segunda linha (AC) $\rightarrow$ (-1,-3) $\rightarrow$ (9,2)

		Linha calculada: 
		
 		x1: -1 ,x2: 9 ,y1: -3 ,y2: 2

		Linha calculada: 
		
		x1: 5 ,x2: 9 ,y1: 0 ,y2: 2

		Linha aceita: 
		
		x1: 5 ,x2: 6 ,y1: 0 ,y2: 0
		
	\end{enumerate}
