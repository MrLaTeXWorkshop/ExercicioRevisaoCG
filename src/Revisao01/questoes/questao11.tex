\section*{Antialiasing}

 	        38) \begin{enumerate}[label=\alph*.]
					\setlength\itemsep{1em}
		
					\item Superamostragem é o algoritmo onde a intensidade do pixel é calculada em uma resolução
					mais alta, para ser visualizada em uma resolução mais baixa
					
					\textbf{Vantagens} :  fácil de implementar
					
					\textbf{Desvantagens} : Força bruta, mais espaço de armazenamento, mais
					tempo de processamento
		
					\item Amostragem por áreas é o algoritmo de intensidade do pixel onde é calculado pelo tamanho da 	área do pixel que é interceptada/sobreposta pelo objeto
					
					\textbf{Vantagens} : 
					
					\begin{itemize}
							\item Quase elimina o aliasing devido à subamostragem;
							\item Permite o pré-cálculo de filtros caros;
							\item Excelente qualidade de imagem;
					\end{itemize}
																 
					\textbf{Desvantagens} : Para formas de amostragem diferentes de polígonos, 
					este algoritmo pode ter um preço computacional maior.
		
					\item Uso de máscaras é um algoritmo onde a intensidade do pixel é calculada a partir
					da distribuição de pesos entre os pixels dentro de uma grade
					
					\textbf{Vantagens} :  \begin{itemize}
						\item Aumenta a resolução da imagem original;
						\item Encontra  as  melhores  posições  dos  pixels  atribuídos.  Este  procedimento reduz  a percepção do efeito escadinha ao retornar uma imagem com melhor qualidade visual;
					\end{itemize}
					
					\textbf{Desvantagens} : Para formas de amostragem diferentes de polígonos, 
					este algoritmo pode ter um preço computacional maior.
					
					\item Pixel Phasing é um algoritmo onde as extremidades são suavizadas, aqui, as posições dos pixels são deslocadas para posições mais aproximadas especificadas pela geometria do objeto. O algortimo também  permite  o  ajuste  de  pixels  individuais 
para  um meio  adicional  ocorrendo  uma  distribuição  de  intensidade
					
					\textbf{Vantagens} :  \begin{itemize}
						\item Aumenta a resolução da imagem original;
						\item Encontra  as  melhores  posições  dos  pixels  atribuídos.  Este  procedimento reduz  a percepção do efeito escadinha ao retornar uma imagem com melhor qualidade visual;
					\end{itemize}
					
					\textbf{Desvantagens} : Para formas de amostragem diferentes de polígonos, 
					este algoritmo pode ter um preço computacional maior.
									
 			  \end{enumerate}
 			  