\section*{Antialiasing}

 	        38) \begin{enumerate}[label=\alph*.]
					\setlength\itemsep{1em}
		
					\item Superamostragem é o algoritmo onde a intensidade do pixel é calculada em uma resolução mais alta, para ser visualizada em uma resolução mais baixa
					
					\textbf{Vantagens} :  fácil de implementar
					
					\textbf{Desvantagens} :
					\begin{itemize}
						\item  Possui um maior gasto de armazenamento, mais
					tempo de processamento e complexidade
						\item Em comparação com os outros, possui uma qualidade ruim
					\end{itemize}
		
					\item Amostragem por áreas é o algoritmo de intensidade do pixel onde é calculado pelo tamanho da área do pixel que é interceptada/sobreposta pelo objeto
					
					\textbf{Vantagens} : 
					
					\begin{itemize}
							\item Qualidade melhor, mais precisa;
					\end{itemize}
																 
					\textbf{Desvantagens} : Preço computacional maior
		
					\item Uso de máscaras é um algoritmo onde é calculado um peso, para indicar o nível
					da cor a ser atribuída ao pixel vizinho correspondente
					
					\textbf{Vantagens} :  
					
					\begin{itemize}
						\item Complexidade mais baixa
					\end{itemize}
					
					\textbf{Desvantagens} : 
					\begin{itemize}
						\item Predefinição dos pesos não representa uma variação
					da cor dos pixels vizinhos gere uma boa aproximação
						\item Não existe um jeito explícito de escolher os pesos dos pixels
					\end{itemize}
					
					\item Pixel Phasing é um algoritmo onde as extremidades são suavizadas, onde as posições dos pixels são deslocadas para posições mais aproximadas especificadas pela geometria do objeto. O algortimo também  permite  o  ajuste  de  pixels  individuais 
para  um meio  adicional  ocorrendo  uma  distribuição  de  intensidade
					
					\textbf{Vantagens} :  \begin{itemize}
						\item Qualidade superior a todos os anteriores
					\end{itemize}
					
					\textbf{Desvantagens} : Depedência de hardware para aplicar esse técnica(Monitor)
									
 			  \end{enumerate}
 			  