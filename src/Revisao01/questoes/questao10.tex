\section*{Preenchimento de Áreas}

  \begin{enumerate}[label=\arabic*)] \addtocounter{enumi}{34}
   
   		\item 
   		
        \begin{enumerate}[label=\alph*.]
		\setlength\itemsep{1em}
		
		\item Boundary fill é o algoritmo usado para preencher pontos com uma cor
		indicada, até encontrar uma cor alvo ou limite, indicando a fronteira da área.

		\textbf{Vantagens} : Lógica simples, além de ser fácil de implementar
		
		\textbf{Desvantagens} : 
		
		\begin{itemize}
			\item A cor da borda deve ser a mesma para todas as arestas do polígono.
			\item Ao usar a análise de conectividade 8(8 vizinhos), com arestas inclinadas, pode
			ocorrer o vazamento, resultando em preenchimento da área externa.
		\end{itemize}
		
		\item Flood Fill é o algoritmo que tem como objetivo recolorir uma determinada área de
		pixels com outra cor, substituindo a cor antiga por uma nova.
	
		\textbf{Vantagens} : 
		
		\begin{itemize}
			\item o preenchimento colore uma área inteira em uma figura fechada por meio de pixels interconectados usando uma única cor.
			\item  É uma maneira fácil de preencher as cores nos gráficos. Um apenas toma a forma e começa o flood fill.
			\item  O algoritmo funciona de forma a dar a todos os pixels dentro do limite a mesma cor
		\end{itemize}

		\textbf{Desvantagens} : não é adequado para desenhar polígonos preenchidos, pois perderá alguns pixels em cantos mais agudos.
		
		\item ScanLine é o algoritmo que processa uma linha por vez, em vez de processar um pixel (um ponto na exibição raster) de cada vez.
		
		\textbf{Vantagens} : 
		\begin{itemize}
			\item Classificar vértices ao longo da normal do plano de varredura reduz o número de comparações entre as bordas
			\item Não é necessário traduzir as coordenadas de todos os vértices da memória principal para a memória de trabalho - apenas os vértices que definem as arestas que cruzam a linha de varredura atual precisam estar na memória ativa, e cada vértice é lido apenas uma vez.
		\end{itemize}
		
		\textbf{Desvantagens} : 
		
				\begin{itemize}
					\item O algoritmo pode ter problemas em desenhar linhas horizontais, onde o
					número de interseções no 'x' é par.
					\item Requer todos os polígonos enviados ao renderizador antes de desenhar.
					\item Possui uma complexidade maior do que os outros algoritmos, já que
					precisam ordenar as vértices a cada iteração
				\end{itemize}

 	\end{enumerate}
 	
 	\item  
 	
 	\item 
 	
 	        \begin{enumerate}[label=\alph*.]
					\setlength\itemsep{1em}
		
					\item Boundary fill 
		
					\item Flood Fill 
		
					\item ScanLine 
									
 			  \end{enumerate}
 			 
   \end{enumerate}
