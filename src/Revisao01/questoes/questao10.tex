\section*{Preenchimento de Áreas}

  \begin{enumerate}[label=\arabic*)] \addtocounter{enumi}{34}
   
   		\item 
   		
        \begin{enumerate}[label=\alph*.]
		\setlength\itemsep{1em}
		
		\item Boundary fill é o algoritmo usado com frequência em computação gráfica para preencher uma cor desejada dentro de um polígono fechado com a mesma cor de limite para todos os seus lados.
		
		\textbf{Vantagens} : Lógica simples e simples de implementar
		
		\textbf{Desvantagens} : A cor da borda deve ser a mesma para todas as arestas do polígono.
		
		\item Flood Fill é o algoritmo que determina e altera a área conectada a um determinado nó em uma matriz multidimensional com algum atributo correspondente.
	
		\textbf{Vantagens} : o preenchimento colore uma área inteira em uma figura fechada por meio de pixels interconectados usando uma única cor.
		
       É uma maneira fácil de preencher as cores nos gráficos. Um apenas toma a forma e começa o flood fill.
       
       O algoritmo funciona de forma a dar a todos os pixels dentro do limite a mesma cor
		
		\textbf{Desvantagens} : não é adequado para desenhar polígonos preenchidos, pois perderá alguns pixels em cantos mais agudos.
		
		\item ScanLine é o algoritmo que processa uma linha por vez, em vez de processar um pixel (um ponto na exibição raster) de cada vez.
		
		\textbf{Vantagens} : 
		
		Classificar vértices ao longo da normal do plano de varredura reduz o número de comparações entre as bordas
		
       Não é necessário traduzir as coordenadas de todos os vértices da memória principal para a memória de trabalho - apenas os vértices que definem as arestas que cruzam a linha de varredura atual precisam estar na memória ativa, e cada vértice é lido apenas uma vez.
       
		\textbf{Desvantagens} : 
		
                    É um algoritmo mais complexo.

                    Requer todos os polígonos enviados ao renderizador antes de desenhar.
									
 	\end{enumerate}
 	
 	\item  
 	
 	\item 
 	
 	        \begin{enumerate}[label=\alph*.]
					\setlength\itemsep{1em}
		
					\item Boundary fill 
		
					\item Flood Fill 
		
					\item ScanLine 
									
 			  \end{enumerate}
 			 
   \end{enumerate}
