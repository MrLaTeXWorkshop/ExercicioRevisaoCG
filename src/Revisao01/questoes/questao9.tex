\section*{Sutherland-Hodgeman}

  \begin{enumerate}[label=\arabic*)]\addtocounter{enumi}{31}
   
   		\item 
   		O algoritmo Sutherland – Hodgman funciona estendendo cada linha do polígono do clipe convexo por vez, e selecionando apenas os vértices do polígono que estão visíveis.
       \item 
 Para a lista de vértices ser atualizada precisamos considerar 4 possibilidades. A
primeira acontece se ambos os vértices estão dentro da área de  clipping, a segunda seria se o
primeiro vértice está dentro e o segundo vértice está fora. A próxima é quando o primeiro vértice
está fora e o segundo dentro e, por último, se ambos estão fora.
       \item 
       
       \begin{itemize}
       	   \item (-1, 1)
	       \item (-1, 6)
  			\item (1, 6)
			\item (5, 4)
			\item (5, 1)
       \end{itemize}

   \end{enumerate}
