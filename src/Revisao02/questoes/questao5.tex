\section*{Superfícies Implícitas}

	\begin{enumerate} \addtocounter{enumi}{12}
		\item  
			\begin{itemize}
				\item Facil implementação
				\item A união ou interseção suave entre superficies implícitas são definidas de modo fácil.
				\item A noção de volume na modelagem implícita facilita a detecção de colisão e assim a 
				construção de objetos complexos;
			\end{itemize}

		\newpage

		\item 
			As formas de representar implícitamente representar uma
			superfície são:
			\begin{itemize}
				\item Algebricamente
				\item Blobby models 
				\item Geração Procedural 
				\item Esqueleto
				\item Amostragem
			\end{itemize}

		\item A utilização de Blobby falicita a representação, pois
		esta forma varia o raio, logo o tamanho do objeto, resultado em
		poucas distorções nas curvaturas dos objetos, com o ganho de um 
		menor gasto de memória para a exibição dos módulos. Já Voxels possuem 
		um único ponto no espaço e tamanho único, dificultando a representação,
		quando comparado a Blobby.
	\end{enumerate}
